\documentclass{beamer}

\batchmode

\usepackage{amsmath,amssymb,enumerate,epsfig,bbm,calc,color,ifthen,capt-of}

\usetheme{Berlin}
\usecolortheme{mit}

%% <language_settings> %%
\usepackage[utf8x]{inputenc}
\usepackage[czech]{babel}
%% </language_settings> %%

%% Source code insertions
\usepackage{verbatim}

%% Include images
\usepackage{graphicx}

%% Hyperlinks
\usepackage{hyperref}

%% <change me> %%
\title{The MIPS Battle: Gentoo vs Others}
\subtitle{FOSDEM 2014}
\author[Markos Chandras \& Panagiotis Christopoulos]{Markos Chandras $<$hwoarang@gentoo.org$>$ \\ Panagiotis Christopoulos $<$pchrist@gentoo.org$>$}
\date{Unknown Date}
%% </change me> %%

%% <logo> %%
\usepackage[absolute,overlay]{textpos}
\setlength{\TPHorizModule}{1mm}
\setlength{\TPVertModule}{1mm}
\newcommand{\MyLogo}{
	\begin{textblock}{14}(110,7)
		\includegraphics[width=14mm,height=14mm]{gentoo-logo.png}
	\end{textblock}
}

\newcommand*\oldmacro{}
\let\oldmacro\insertshorttitle
\renewcommand*\insertshorttitle{
	\MyLogo
	\oldmacro\hfill
}
%% </logo> %%

%% <bgimage> %%
\usebackgroundtemplate{
	\includegraphics[width=\paperwidth,height=\paperheight]{gentoo-background-1.png}
}
%% </bgimage> %%


\begin{document}

% ----------------------------------- Introduction -------------------------------------
\frame{\titlepage}

\section{Introduction}

\subsection{Do you want to know more about Markos?}

\begin{frame}{Markos Chandras}
	\begin{itemize}
		\item Gentoo Developer since 2009
		\item MIPS, Release Engineering, Embedded, Desktop/LXDE, Community Relations, Recruiter and others.
		\item Works at Imagination Technologies, mainly involved in MIPS Linux Kernel maintenance.
	\end{itemize}
	Works closely with the Release Engineering Team to provide up-to-date stage3 tarballs for all the possible MIPS variants (mips1, mips2, mips3, mips4, etc)
\end{frame}

\subsection{..and even more about Panagiotis?}
\begin{frame}{Panagiotis Christopoulos}
% TODO
\end{frame}
% -----------------------------------------------------------------------------

% ----------------------------------- Overview -------------------------------------
\section{Overview of the Gentoo/MIPS Port}

\subsection{History}

\begin{frame}{History}
	\begin{itemize}
		\item O32 only until 2010.
		\item Active port until 2008 where its lead developers left Gentoo.
		\item Matt Turner $<$mattst88@gentoo.org$>$ joined in 2010 and added n32 and n64 multilib support. He also added new MIPS variants to catalyst (release engineering automated tool) and built stages for all the MIPS
	variants. The MIPS port started to look alive again.
		\item Anthony Basile $<$blueness@gentoo.org$>$ joined in 2012. He started delivering uClibc MIPS stages.
	\end{itemize}
\end{frame}

\subsection{The Gentoo/MIPS port in 2013}

\begin{frame}{Features}
	\begin{itemize}
		\item All MIPS ISAs supported (mips1, mips2, mips3, mips4, mips32, mips32r2, mips64, mips64r2)
		\item Little and Big Endian
		\item All 3 ABIs supported in full multilib environment.
		\item Single ABI variants are also supported.
		\item glibc and uClibc stages (musl support ETA 2014)
		\item stage4 for loongson 2F laptops (3A support ETA 2014). MIPS is the only Gentoo arch providing stage4 to users.
		\item Reasonably good package coverage (2262 packages, 13.31\% of total packages in Gentoo portage tree (src: GMN Nov 2013))
	\end{itemize}
\end{frame}

\subsection{Limitations}

\begin{frame}{But we are not perfect...}
	\begin{itemize}
		\item Testing only architecture. No Arch Testers, strong QA control, no stable support.
		\item Limited runtime testing of stages.
		\item Limited available hardware for porting purposes.
		\item Limited manpower.
	\end{itemize}
\end{frame}


% ----------------------------------- Other Distributions -------------------------------------
\section{Lets Fight{!}}

\subsection{Debian}
\begin{frame}{... vs Debian}
+1 for Debian
	\begin{itemize}
		\item Excellent package coverage.
		\item Wider testing. Need to match the Debian release standards.
		\item Better QA control.
	\end{itemize}
-1 for Debian
	\begin{itemize}
		\item No official MIPS64 bit support. MIPS64el/n64 GSOC 2013 earlier this year.
		\item No MIPS64 Big-Endian yet.
		\item No alternative libc for root file systems.
		\item Single ISA supported for 32-bit/64-bit. mips2 for mips/mipsel, mips64r2 (which likely to be downgraded to MIPSIII) for mips64el.
	\end{itemize}
Limited hardware and manpower for both distributions.
\end{frame}
\end{document}
